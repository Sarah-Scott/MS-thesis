\chapter{Background}



\section{TPM 2.0}

A Trusted Platform Module (TPM) is a microcontroller that complies with the ISO/IEC 11889:2015 international standard.
The TPM and its specification were designed by the Trusted Computing Group (TCG) to act as a hardware anchor for PC system security \cite{{PracticalGuide}}. To this end, TPMs have the abilities necessary for secure generation of keys, algorithm agility, secure storage of keys, enhanced authorization, device health attestation, device identification, NVRAM storage and more. 

The TPM's key generator is based on its own random number generator (RNG) so that it does not rely on external sources of randomness. These keys can be used for a multitude of purposes and may be created or destroyed as often as needed. Due to algorithm agility, the TPM can use nearly any cryptographic algorithm. As a result, keys may utilize asymmetric algorithms such as RSA or ECC, or they may utilize symmetric algorithms such as AES or DES. Additionally, a variety of key strengths (i.e., key sizes) and hash algorithms may be used. By design, keys stored within the TPM are protected against software attacks. Keys may optionally be further protected using enhanced authorization. Enhanced authorization (EA) allows a key or other TPM entity to be authorized using a password, HMAC, or policy. This flexibility allows for varying complexities in the requirements for accessing an entity. 


Device health attestation data provided by a TPM offers cryptographic proof of software state. Attestation data comes in the form of a quote which is a signed hash over a selection of platform configuration registers. Platform configuration registers (PCRs) store the results from a chain of boot time measurements in a way that guarantees integrity. In particular, a PCR cannot be rolled back to a previous value resulting in a measurement being undone. To tie attestation data to a specific device, the key that performed the signing operation must be cryptographically bound to that device, that is, the key must be a \textit{secure device identifier}. There exist a variety of additional applications, such as network authentication, which similarly require a device to be definitively identified. Public key certificates may be distributed to remote entities in order to prove that the corresponding private key resides in a specific TPM-containing device. These certificates are stored within the TPM's NVRAM providing protection from accidental erasure in the scenario that the device's hard drive gets wiped. 






\subsection{Keys}



Since the focus of this paper is on the usage of TPM keys for secure device identifiers, the discussion on cryptographic key in this section is limited to TPM keys which utilize asymmetric cryptography.  A key may then be created using one of two commands.
\begin{itemize}
  \item \verb|TPM2_CreatePrimary|: A Primary key is produced based on the current Primary Seed. A Primary key may be persisted within the TPM. Otherwise it must be recreated after a TPM reset.
  \item \verb|TPM2_Create|: An Ordinary key is produced based on a seed taken from the RNG. An Ordinary key is the child of another key; it is wrapped by that parent key. It may be persisted within the TPM or persisted external to the TPM in the form of an encrypted key blob. The blob is only loadable using the parent key's authorization in the TPM that created it.
\end{itemize}
Keys have attributes that are set at creation-time. These attributes are permanent and include the following: \verb|FixedTPM|, \verb|Sign|, \verb|Decrypt|, and \verb|Restricted|. The \verb|FixedTPM| attribute indicates that the private key cannot be duplicated. A key pair with the \verb|Sign| attribute set consists of a private signing key and a public signature-verification key. When properly handled, private signing keys can provide integrity, authenticity, and nonrepudiation. A key pair with the \verb|Decrypt| attribute set consists of a public encryption key and a private decryption key. When properly handled, public encryption keys can provide confidentiality. A key with both the \verb|Sign| and \verb|Decrypt| attributes set is called a Combined key. US NIST SP800-57 disallows the use of Combined keys for the reason that it may weaken the security guarantees associated with one or both of the attributes. Furthermore, a key pair may have the \verb|Restricted| attribute set, limiting the operations of the private key to TPM generated data.

The Coq model inductively defines a \verb|pubKey| and \verb|privKey| type for public keys and private keys respectively. A key of either of these types requires a unique identifier and a sequence of boolean values describing whether a particular attribute is set or not set. A key pair consists of a \verb|pubKey| and a \verb|privKey| with the same identifier and attributes. The model does not differentiate between Primary and Ordinary keys.
\begin{figure}[h]
\begin{lstlisting}[language=Coq]
Inductive pubKey : Type :=
| Public : keyIdType -> Restricted -> Sign -> Decrypt -> FixedTPM -> pubKey.

Inductive privKey : Type :=
| Private : keyIdType -> Restricted -> Sign -> Decrypt -> FixedTPM -> privKey.
\end{lstlisting}

\caption{Model of Keys}
\end{figure}


%The \verb|Restricted| attribute can provide important security implications. A restricted signing key may only sign a digest that has been produced by the TPM. Enforcement of this constraint is reliant on a 4-byte magic value called \verb|TPM_Generated| [\cite{TPMSpec}]. All structures that the TPM constructs from internal data begins with this value. Such structures include keys, platform configuration registers (PCRs), and audit digests. These structures contribute to two primary use cases for restricted signing keys: (1) key certification and (2) attestation. Use case 1 proves that a new key resides in the same TPM as some known restricted key. Use case 2 utilizes the \verb|Restricted| attribute to provide assurance that a signature over PCRs or audit logs was in fact over a digest generated by that particular TPM. Additionally, a restricted signing key can sign data supplied to the TPM externally by using the \verb|TPM2_Hash| command. In this case the \verb|TPM2_Hash| command produces a ticket asserting that the TPM itself calculated this hash and will later sign it. A restricted signing key will not sign external data without this ticket. To prevent spoofing of another TPM's internal data as one's own, the \verb|TPM2_Hash| command will only produce a ticket if the external data does not begin with the \verb|TPM_Generated| value.

The \verb|Restricted| attribute provides important security implications. A restricted signing key may only sign a digest that has been produced by the TPM. Enforcement of this constraint is reliant on a 4-byte magic value called \verb|TPM_Generated| \cite{TPMSpec}. All structures that the TPM constructs from internal data begins with this value. Such structures include keys and PCRs. These structures result in several interesting and significant uses for restricted signing keys, namely key certification and attestation. In particular, a restricted signing key is used during key certification to prove that a new key resides in the same TPM as itself. And during attestation, a restricted signing key is used to prove that a quote is the result of the PCR values within the same TPM in which the key itself resides. Additionally, a restricted signing key has the ability to sign data supplied to the TPM externally by using the \verb|TPM2_Hash| command. When used for this purpose, the hash operation is called a signature hash. The \verb|TPM2_Hash| command produces a ticket asserting that the TPM itself calculated this hash and will later sign it. A restricted signing key will not sign external data without this ticket. To prevent spoofing of another TPM's internal data as one's own, the \verb|TPM2_Hash| command only produces a ticket if the external data does not begin with the \verb|TPM_Generated| value.


A restricted decryption key is called a storage key. Only storage keys can be used as parents to create or load child objects or to activate credentials \cite{PracticalGuide}. All TPMs are shipped with an essential storage key: the endorsement key. The endorsement key (EK) is installed by the TPM Manufacturer and stored in a shielded location on the TPM. The corresponding EK certificate serves a significant role in the enrollment of secure device identifiers. This process will be discussed in further detail in later sections.

A certificate contains a public key and an identity and is signed by a trusted Certificate Authority. A certificate binds a public key to an identity. The term certificate specifically refers to an X.509 v3 digital certificate. 
The EK certificate includes the public part of the EK itself as well as various assertions regarding the security qualities and provenance of the TPM \cite{EKSpec}. The EK certificate binds the EK to a specific TPM. For keys created by entities other than the TPM manufacturer (i.e., the OEM and the Owner), a certificate's identity field contains non-TPM device information. This information should be globally unique per device \cite{DevIDSpec-IEEE}. In this model, certificates are abstractly defined as the inductive \verb|signedCert| type. A \verb|signedCert| requires a public key, an identifier, and a private key. An identifier may include include information describing either the TPM or the device. The private key parameter denotes the key which performed the signature over the certificate.
\begin{figure}[h]
\begin{lstlisting}[language=Coq]
Inductive identifier : Type :=
| TPM_info : tpmInfoType -> identifier
| Device_info : deviceInfoType -> identifier.

Inductive signedCert : Type :=
| Cert : pubKey -> identifier -> privKey -> signedCert.
\end{lstlisting}
\caption{Model of Certificates}
\end{figure}





\section{Logical Foundation of Coq}



The Calculus of Inductive Constructions (CIC) is the underlying formalism of the interactive proof assistant Coq \cite{CIC}. This section shall not discuss this logical system in great detail and instead focuses only on the aspects which are relevant to this work. In particular, this section examines the sort \verb|Prop| and the interesting ways in which it is utilized. The sort \verb|Prop| is the universe of logical propositions. Inductive definitions may be used together with the sort \verb|Prop| to define relations. These relations are called inductive propositions. Some inductive propositions are defined in Coq's standard library while others must be defined by the user. One simple yet important inductive proposition in Coq's standard library is the definition of equality. 
\begin{figure}[h]
\begin{lstlisting}[language=Coq]
Inductive eq {A : Type} (x : A) : A -> Prop :=
| eq_refl : eq x x.
\end{lstlisting}
\caption{Equality in Coq}
\end{figure}
Equality has only one constructor, namely \verb|eq_refl| which corresponds with reflexivity. The constructors of an inductive proposition act as introduction rules; for an element to satisfy a particular inductive proposition, it must be built from its constructors. Using \verb|eq|'s single constructor, one can prove nearly all of the fundamental properties of equality (e.g., symmetry and transitivity), though there is one important property of equality which cannot be proven to hold in general, that is, decidability. 
A type has decidable equality if any two elements of that type are either equal or not equal. Due to the intuitionistic nature of CIC, not all types in Coq have this property. Intuitionistic logic differs from classical logic in that it rejects the law of excluded middle and double negation elimination. Therefore, decidability is not guaranteed over any proposition: equality included. If decidable equality over a specific type is needed, one must declare that type an instance of the \verb|DecEq| class and explicitly prove that this property holds.

%These logical propositions have some surprising characteristics due to the intuitionistic nature of CIC. Intuitionistic logic differs from classical logic in that it rejects the law of excluded middle and double negation elimination. This fundamental difference means that many propositions which mathematicians and computer scientists take for granted are no longer provable. One notable instance of such a proposition is regarding equality. We have always learned that for any two items $a$ and $b$ that either $a = b$ or $a \neq b$. But, since Coq does not assume the law of excluded middle, this simply is not always the case. In fact, if one wishes to use this property,
