\chapter{Conclusion}


\section{Conclusion}

% CONCLUDE


It is important in secure systems to have strong identification mechanisms in place. By definition, a DevID should fulfill this demand. Utilizing formal methods to analyze DevIDs is a crucial step in affirming their reliability and veracity. In this way, more trust may be put into the real-life implementations of these secure systems.
The contributions of the research presented in this thesis may be summarized as (1) a general investigation into TPM-based DevIDs and chains of certificates, (2) the design and implementation of an abstract formal model of command execution, (3) an in-depth analysis of two TCG-provided key certification protocols, and (4) the discovery of a potential shortcoming of the IAK provisioning procedure and consequently a recommendation for its improvement.



\section{Future Work}


Although the modeled command library and execution environment are useful in verifying properties over some key certification protocols, further improvements are necessary to extend the applicability of this model to a broader range of problems. An initial step towards achieving this is to expand the command library to include the TPM commands necessary for modeling the attestation variation of these provisioning protocols. These attestation variations use PRCs to inform a remote CA of the internal state of a certificate-requesting entity. This variation is strongly encouraged to be performed during certification of an IAK so that the CA may be assured it is issuing a certificate to a device running trusted software. To include PCR-related commands, the structure of abstract state should be modified from a pair to a record similar to that of the work of Halling et al. \cite{PrivacyCAAnalysis-Hall}. These advancements to the model also provide the mechanisms necessary to verify the changes to the IAK provisioning procedure recommended in this thesis. In addition to adding those particular commands, it is useful to include more TPM and TSS commands in general. By expanding the range of commands supported by this model, we can describe a broad range of situations, even those unrelated to DevID provisioning protocols.  Besides expanding the modeled TPM command library, we may also improve on the execution environment. Specifically, more complex control sequences such as branching and looping should be included so that we can describe more elaborate scenarios.  In conclusion, this research has revealed several areas for further exploration and improvement. These future areas of work will enhance the capabilities of this model and provide a formal system for verification of TPM-related properties.
