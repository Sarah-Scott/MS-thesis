\begin{abstractlong}

A secure device identifier (DevID) is defined as an identifier that is cryptographically bound to a device. TPM keys are an ideal choice for implementation of these identifiers.
The TCG's specification ``TPM 2.0 Keys for Device Identity and Attestation'' describes several procedures for remotely proving a key to be resident in a specific device's TPM and thereby satisfying the requirements for a DevID. These procedures are carefully constructed protocols that are intended to be performed by a trusted Certificate Authority (CA) in communication with a certificate-requesting device. DevID certificates provisioned by an OEM at device manufacturing time should provide definitive evidence that a key belongs to a specific device. Whereas DevID certificates provisioned by a device owner
require a chain of certificates to prove that a key belongs to a specific device. This distinction is due to the differences in the respective protocols prescribed by the TCG's specification. \textit{I aim to abstractly model these key certification protocols and formally verify their resulting assurances}. I choose this goal since the TCG themselves do not provide proofs or clear justifications for how the protocols provide these assurances. 

\end{abstractlong}





\begin{acknowledgementslong}

I would like to thank my husband for his endless support. I believe he now knows as much as I do on this topic.

\end{acknowledgementslong}

