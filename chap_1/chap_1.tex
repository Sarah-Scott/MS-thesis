\chapter{Introduction}

% RELATED WORKS SECTION


Development and deployment of trusted systems often require definitive identification of devices. A remote entity should have confidence that a device is as it claims to be. An ideal method for fulfulling this need is through the utilization of TPM keys as secure device identitifiers. A secure device identifier (DevID) is defined as an identifier that is cryptographically bound to a device \cite{DevIDSpec-IEEE}. 
A DevID must not be transferable from one device to another as that would allow distinct devices to be identified as the same. 
Since the Trusted Platform Module (TPM) is a secure Root of Trust for Storage, it provides the necessary protections for storing these identifiers and enforcing this constraint. 

The TCG's specification "TPM 2.0 Keys for Device Identity and Attestation" describes several methods for remotely proving a key to be resident in a specific device's TPM. These methods are carefully constructed protocols which are intended to be performed by a trusted Certificate Authority (CA) in communication with a certificate-requesting device. These protocols are designed to maintain a cryptographic evidentiary chain linking a DevID to a specific TPM \cite{DevIDSpec-TCG}. 
DevID certificates provisioned by an OEM at device manufacturing time provide definitive evidence that a key belongs to a specific device. Whereas DevID certificates provisioned by a device owner require a chain of certificates to prove that a key belongs to a specific device. This distinction is due to differences in the respective protocols prescribed by the TCG's specification. 
For each provisioning protocol described in the specification, the TCG outlines steps for the CA and the certificate-requesting entity to perform. Furthermore, the TCG claims that each protocol provides certain assurances. These assurances are the basis for the resulting cryptographic evidentiary chain.
Each assurance manifests as an assertion regarding either TPM-residency, key attributes, or previously-issued certificates.

This work places special emphasis on two of the DevID-provisioning protocols, namely OEM Creation of an IAK Certificate based on an EK Certificate and Owner Creation of an LAK Certificate based on an IAK Certificate. I select these protocols due to their especially significant security implications. The primary goal of this work is to abstractly model these two protocols and formally verify their resulting assurances. I choose this goal since the TCG themselves do not provide any proofs or clear justifications for how the protocols might provide these assurances. 




% related works

\section{Related Works}



My work has many parallels to the work of Halling et al. \cite{TPM12Model,PrivacyCAAnalysis}.
They abstractly model a large subset of the TPM 1.2 commands in the PVS specification language. TPM command execution is modeled as a transition system over an abstract system state. I implement this same technique in my own work but over a small subset of TPM 2.0 commands as well as several non-TPM commands. Their work attempts verification of the Privacy CA protocol. In their verification, they consider the functional correctness of the protocol and specifically examine the protocol implementation to ensure that it produces the structure that it should. They do not consider specific attacks on this structure. Similarly, my work attempts verification of protocols involving a certificate authority and focuses on determining whether the protocol implementations result in the assurances that they should provide.

The work of Delaune et al. uses the ProVerif tool to examine  Authentication and PCRs in the TPM 1.2 
\cite{AuthAnalysis,PCRAnalysis}.
Their work focuses on verifying cryptographic properties over these elements of the TPM.


3,4
Related Works
\cite{DAAAnalysis} 
Tamarin prover

\cite{EAAnalysis} 
stateful applied pi calculus
SAPIC tool and Tamarin prover

\cite{HMACAnalysis}



