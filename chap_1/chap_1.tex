\chapter{Introduction}

% maybe expand intro

% ****related works section*******

Development and deployment of trusted systems often require definitive identification of devices. A remote entity should have confidence that a device is as it claims to be. An ideal method for fulfulling this need is through the utilization of TPM keys as secure device identitifiers. A secure device identifier (DevID) is defined as an identifier that is cryptographically bound to a device \cite{DevIDSpec-IEEE}. 
A DevID must not be transferable from one device to another as that would allow distinct devices to be identified as the same. 
Since the Trusted Platform Module (TPM) is a secure Root of Trust for Storage, it provides the necessary protections for storing these identifiers and enforcing this constraint. 

The TCG's specification "TPM 2.0 Keys for Device Identity and Attestation" describes several methods for remotely proving a key to be resident in a specific device's TPM. These methods are carefully constructed protocols which are intended to be performed by a trusted Certificate Authority (CA) in communication with a certificate-requesting device. These protocols are designed to maintain a cryptographic evidentiary chain linking a DevID to a specific TPM \cite{DevIDSpec-TCG}. 
DevID certificates provisioned by an OEM at device manufacturing time provide definitive evidence that a key belongs to a specific device. Whereas DevID certificates provisioned by a device owner require a chain of certificates to prove that a key belongs to a specific device. This distinction is due to differences in the respective protocols prescribed by the TCG's specification. 
For each provisioning protocol described in the specification, the TCG outlines steps for the CA and the certificate-requesting entity to perform. Furthermore, the TCG claims that each protocol provides certain assurances. These assurances are the basis for the resulting cryptographic evidentiary chain.
Each assurance manifests as an assertion regarding either TPM-residency, key attributes, or previously-issued certificates.

This work places special emphasis on two of the DevID-provisioning protocols, namely OEM Creation of an IAK Certificate based on an EK Certificate and Owner Creation of an LAK Certificate based on an IAK Certificate. I select these protocols due to their especially significant security implications. The primary goal of this work is to abstractly model these two protocols and formally verify their resulting assurances. I choose this goal since the TCG themselves do not provide any proofs or clear justifications for how the protocols might provide these assurances. 




% related works

\section{Related Works}
Related Works
\cite{DAAAnalysis}
\cite{TPM12Model}
\cite{PrivacyCAAnalysis}
\cite{AuthAnalysis}
\cite{PCRAnalysis}
\cite{EAAnalysis}
\cite{HMACAnalysis}



