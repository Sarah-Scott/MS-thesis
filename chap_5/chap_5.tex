\chapter{Identity Provisioning}


To mantain a cryptographic evidentiary chain linking a DevID to a specific TPM and device, the CA should follow certain provisioning procedures. The TCG describes several such procedures in their specification ``TPM 2.0 Keys for Device Identity and Attestation'' \cite{DevIDSpec-TCG}. 
This chapter considers two of these procedures in detail: OEM creation of an IAK certificate based on an EK certificate and Owner creation of an LAK certificate based on an IAK certificate. I select these two protocols since they bear the most significance in enrollment of additional DevIDs (recall that, in a chain of certificates, IAK certificates may be root nodes and LAK certificates may be parent nodes). For each protocol, the specification outlines steps for the CA and the certificate-requesting entity to perform.
Furthermore, the specification claims that each protocol provides certain assurances. Each assurance manifests as an assertion regarding either TPM-residency, key attributes, or previously-issued certificates and provides the basis for the resulting cryptographic evidentiary chain formed by a chain of certificates. Therefore, it is of utmost importance to verify that each protocol guarantees its associated assurances.
Since the TCG themselves do not provide proofs or clear justifications to support these claims, \textit{the goal of this work is to abstractly model these two protocols and formally verify their resulting assurances}.


I model these protocols within Coq's \verb|Module Type| mechanism. This mechanisms allows for the inclusion of parameters that provides the necessary flexibility to describe each protocol generally. Additionally, this mechanism allows for axioms to be defined. When instantiating a \verb|Module Type|, one must provide concrete values for all parameters and prove that all axioms hold.

In conducting these verifications, I consider two scenarios: (1) the certificate-requesting entity and the CA are both trusted to execute their steps correctly and (2) only the CA is trusted to execute its steps correctly. The specification does not state which of these assumptions they reason under. 
Because trust for creation of a new certificate is based on an existing certificate, these scenarios include the presupposition that previously-issued certificates imply the associated assurances of their provisioning protocols.
%The latter scenario clearly leads to a much stronger guarantee, but we still 
For convenience and clarity, this chapter inspects each of these protocols in the reverse order that their dependencies entail. I proceed with this ordering because the LAK provisioning procedure is approximately contained within the IAK provisioning procedure. 
%and we wish to first inspect the simpler protocol.



\section{Owner Creation of LAK Certificate based on IAK Certificate}

%The TCG's specification claims that the following procedure guarantees that the new LAK is  loaded on the same TPM as the IAK.
The TCG's specification claims that the following procedure provides these assurances: (A) the LAK has good attributes and (B) the LAK is  loaded on the same TPM as the IAK. These assurances correspond exactly with the Chain of Trust line C in Figure \ref{fig:cert_rel}. 
\begin{enumerate}[itemsep=0pt,parsep=0pt,partopsep=0pt]
  \setcounter{enumi}{-1}
  \item The Owner creates and loads the LAK
  \item The Owner certifies the LAK with the IAK
  \item The Owner builds the CSR containing:
  \begin{enumerate}[topsep=0pt, itemsep=0pt,parsep=0pt,partopsep=0pt]
    \item The signed \verb|TPM2B_Attest| structure
    \item The IAK certificate
  \end{enumerate}
  \item The Owner takes a signature hash of the CSR
  \item The Owner signs the resulting hash digest with the LAK
  \item The Owner sends the CSR paired with the signed hash to the CA
  \item The CA verifies the recieved data by checking:
  \begin{enumerate}[topsep=0pt, itemsep=0pt,parsep=0pt,partopsep=0pt]
    \item The hash digest against the CSR
    \item The signature on the hash digest with the public LAK
    \item The signature on the \verb|TPM2B_Attest| structure with the public IAK
    \item The signature on the IAK certificate with the public key of the OEM's CA
    \item The attributes of the LAK
  \end{enumerate}
  \item If all of the checks succeed, the CA issues the LAK certificate to the Owner
\end{enumerate}


Modeling this protocol begins by defining parameters for each the Owner and the CA. These parameters correspond with the elements required by the Owner and the CA to perform their respective parts of the procedure. However, elements intended to be received during the communication phases of the protocol are excluded. Specifically, these parameters intend to represent the elements that must be known by each entity \textit{prior} to the start of the protocol. Therefore, the Owner has its LAK, IAK, and IAK certificate, and the CA has its own key and the public key of the OEM's CA. The parameters only explicitly include the public key values of those listed keys pairs. Private key values are computed by taking the inverse of the corresponding public key; these values are stored in the \verb|privLAK|, \verb|privIAK|, and \verb|privCA| variables. To enforce the randomness of cryptographic keys, I define an axiom that requires all key parameters to be pairwise distinct. 
\begin{figure}[h]
\begin{lstlisting}[language=Coq]
(* Owner parameters *)
Parameter pubLAK : pubKey.
Parameter pubIAK : pubKey.
Parameter certIAK : signedCert.

(* CA parameters *)
Parameter pubCA : pubKey.
Parameter pubOEM : pubKey.

(* All keys are pairwise distinct *)
Axiom keys_distinct :
  pubLAK <> pubIAK /\
  pubLAK <> pubCA /\
  pubLAK <> pubOEM /\
  pubIAK <> pubCA /\
  pubIAK <> pubOEM /\
  pubCA <> pubOEM.
\end{lstlisting}
\caption{Parameters of LAK Provisioning Procedure}
\end{figure}

 Similar to how the parameters are separated by ownership, the procedure itself may be separated as well. That is, the procedure may be regarded as the composition of two parts: the Owner's steps (i.e., Steps 0-5) followed by the CA's steps (i.e., Steps 6-7).
 With that in mind, each part of the procedure may be abstractly modeled using the parameters defined above and the sequential command construction defined in Chapter 4. 
\begin{figure}[h!]
\begin{lstlisting}[language=Coq]
Definition steps1to5_Owner : sequence :=
TPM2_Certify 
   pubLAK 
   privIAK ;;
MakeCSR_LDevID 
  (signature (TPM2B_Attest pubLAK) privIAK) 
   certIAK ;;
TPM2_Hash 
  (TCG_CSR_LDevID (signature (TPM2B_Attest pubLAK) privIAK) certIAK) ;;
TPM2_Sign 
  (hash (TCG_CSR_LDevID (signature (TPM2B_Attest pubLAK) privIAK) certIAK)) 
   privLAK ;;
MakePair 
  (TCG_CSR_LDevID (signature (TPM2B_Attest pubLAK) privIAK) certIAK) 
  (signature (hash (TCG_CSR_LDevID (signature (TPM2B_Attest pubLAK) privIAK) certIAK)) privLAK) ;;
Done. 


Definition steps_CA (msg : message) (iak lak : pubKey) (cert : signedCert) : Prop :=
  match msg with
  | (pair (TCG_CSR_LDevID (signature (TPM2B_Attest k) k0') (Cert k0 id k_ca')) (signature m k')) =>
        iak = k0 /\ lak = k /\ cert = (Cert k0 id k_ca') /\
        seq_execute   (iniTPM_CA, inferFrom msg ++ ini_CA)
                      (CheckHash 
                          m
                         (TCG_CSR_LDevID (signature (TPM2B_Attest k) k0') (Cert k0 id k_ca')) ;;
                       CheckSig
                         (signature m k') 
                          k ;;
                       CheckSig 
                         (signature (TPM2B_Attest k) k0') 
                          k0 ;;
                       CheckCert 
                         (Cert k0 id k_ca') 
                          pubOEM ;;
                       CheckAttributes 
                          k 
                          Restricting Signing NonDecrypting Fixing ;;
                       Done)
                      (iniTPM_CA, inferFrom msg ++ ini_CA)
  | _ => False
  end.
\end{lstlisting}
\caption{Model of LAK Provisioning Procedure}
\label{fig:lak_model}
\end{figure}
I construct an object of \verb|sequence| type for the Owner and an object of function type for the CA. Constructing the Owner's steps is straightforward. Since this model disallows the arbitrary creation of keys, Step 0 is assumed to have been performed prior; the results of Step 0 are in fact already encapsulated in the \verb|pubLAK| parameter and \verb|privLAK| variable. Then each remaining step of the Owner corresponds with exactly one command in the model, namely \verb|TPM2_Certify| for Step 1, \verb|MakeCSR_LDevID| for Step 2, \verb|TPM2_Hash| for Step 3, \verb|TPM2_Sign| for Step 4, and \verb|MakePair| for Step 5. 
Constructing the CA's steps is more complex as it relies on external input (i.e., the certification request produced by the Owner's steps). Although this complexity leads to a convoluted function, there is still a straightforward correspondence between the function definition and the real-life procedure. First, the CA waits to receive a certification request from the Owner (see the \verb|msg| input). The request must be in a specific format to be considered valid (see the match statement on \verb|msg|). The CA then executes Step 6 of the procedure (see the sequence within \verb|seq_execute|). If execution succeeds, the CA issues the LAK certificate to the Owner (see the \verb|Prop| return type). I include several additional parameters and criteria to serve as a method for referencing certain elements of the certification request within proof statements. These definitions provide the framework necessary for describing the conditions of each scenario. 


Using only the CA's function, it is trivial to prove Assurance A. The \verb|CheckAttributes| \verb|k Restricting Signing NonDecrypting Fixing| command in the CA's function corresponds with Step 6e of the provisioning procedure 
(see that the CA's function binds the LAK to the variable \verb|k|). Then it is clear to see that successful execution of this command directly implies that the LAK has all of the attributes required to be an attestation key.

Let us attempt formal verification of Assurance B first under the conditions of scenario 1: the Owner and the CA are both trusted to execute their steps correctly. 
%Recall first that the goal is to show that this provisioning procedure guarantees that the new LAK is  loaded on the same TPM as the IAK. Recall that we trust that the IAK provisioning procedure guarantees its associated assurances, primarily that the IAK has good attributes. 
Recall that this verification trusts that the previously-performed IAK provisioning procedure guarantees its associated assurances --- this verification specifically uses the assertion that the IAK has good attributes.
I begin by examining the Owner's steps and its requirements. These requirements can be quantitatively described by a minimal initial state pair. Given a sequence, a minimal initial state pair is defined as the smallest \verb|tpm_state| and \verb|state| that allows for successful execution of the sequence. The proof statement describing this property is constructed by two parts: (1) the minimal initial state is a lower bound on the set of possible initial states and (2) the minimal initial state is sufficient for successful execution. I build a minimal initial state pair for \verb|steps1to5_Owner| using the following intuition: (i) the private LAK and private IAK are loaded on the same TPM because the LAK is certified by the IAK and (ii) the IAK certificate is known to the Owner because it is included in the CSR.  
The proof of the lower bound property uses this exact intuition.
On the other hand, the proof of the sufficiency property uses the constructors of the \verb|execute| relation to demonstrate that there exists a final state pair which satisfies the \verb|seq_execute| relation for the minimal initial state pair and \verb|steps1to5_Owner|. This proof relies on two assumptions, namely that the IAK has good attributes and that the CA issues the LAK certificate.
\begin{figure}[h]
\begin{lstlisting}[language=Coq]
Definition iniTPM_Owner : tpm_state :=
[ privateKey privLAK ;
  privateKey privIAK ].

Definition ini_Owner : state :=
[ signedCertificate certIAK ].

Lemma ini_Owner_lowerBound : forall iniTPM ini fin,
  seq_execute (iniTPM, ini) steps1to5_Owner fin ->
  (iniTPM_Owner \subsetOf iniTPM) /\
  (ini_Owner \subsetOf ini).

Lemma ini_Owner_sufficient : forall msg,
  attestationKey pubIAK ->
  steps_CA msg pubIAK pubLAK certIAK ->
  exists fin, seq_execute (iniTPM_Owner, ini_Owner) steps1to5_Owner fin.
\end{lstlisting}
\caption{Minimal Initial State of Owner}
\end{figure}
 This analysis of the Owner's steps' requirements in the form of a minimal initial state conveniently leads to the conclusion that the LAK and IAK must be loaded on the same TPM for the Owner to execute its steps of the procedure. This conclusion is manifested in the \verb|iniTPM_Owner| variable which contains both the private LAK and private IAK. In conclusion, we have now confirmed that Assurance B is in fact guaranteed by the protocol when we assume that both the Owner and the CA are trusted to execute their steps correctly. 
%That is because the CA verifies the attributes on the LAK (see Appendix X for statement and proof). Since we have now shown Assurance A holds under case 1 and Assurance B holds under both cases 1 and 2, let us now attempt to prove the final objective that Assurance A holds under case 2. 

Let us now attempt formal verification of this same goal under the conditions of scenario 2: only the CA is trusted to execute its steps correctly. This proof is troublesome and likely impossible if we make no assumptions regarding the Owner, but since the certification request and its contents must have been produced by some entity, I consider the Owner to be this entity.
To this end, I describe the Owner and its characteristics as a series of assumptions: the Owner executes some unknown sequence of commands \verb|s|, this sequence produces some message \verb|msg| in the Owner's final \verb|state|, the Owner's initial \verb|tpm_state| may only contain private keys, the Owner's initial \verb|state| may only contain public keys and certificates, and the CA executes its prescribed sequence on the message \verb|msg|. 
\begin{figure}[h]
\begin{lstlisting}[language=Coq]
Theorem lak_and_iak_in_TPM : forall s iniTPM ini finTPM fin msg iak lak cert,
  seq_execute (iniTPM, ini) s (finTPM, fin) -> 
  In msg fin ->
  (forall m', needsGeneratedTPM m' -> ~ In m' iniTPM) ->
  (forall m', needsGenerated m' -> ~ In m' ini) ->
  steps_CA msg iak lak cert ->
  In (privateKey (pubToPrivKey lak)) iniTPM /\ In (privateKey (pubToPrivKey iak)) iniTPM
\end{lstlisting}
\caption{Final Verification Goal for LAK Provisioning Procedure}
\label{fig:lak_goal}
\end{figure}
I argue that these assumptions are reasonable and do not corrupt the conditions regarding the trustworthiness of the Owner. It is important to note that the LAK, IAK, and IAK certificate are universally quantified in these assumptions and make no reference to the Owner's parameters. In particular, these assumptions aim only to constrain the production of the certification request (i.e., \verb|msg|) and its contents to the Owner. 
The restrictions on the Owner's initial state pair are the main contributors to enforcement of this constraint. Due to the nature of command execution, the certification request and all of its contents must have been either produced by a command or contained in the inital state pair. 
Therefore, the restrictions placed on the initial state pair allow only exactly those messages which are impossible to be generated by a command sequence.
The \verb|needsGeneratedTPM| function restricts the Owner's initial \verb|tpm_state| to the inclusion of previously created private keys. While the \verb|needsGenerated| function restricts the Owner's initial \verb|state| to the inclusion of public keys as well as previously issued certificates --- the subject of these certificates may be the Owner itself or any other entity.
Although realistically the Owner has many other messages in its knowledge, these restrictions simply aim to disallow them from being used to build the certification request. 



The next step is to use this series of initial assumptions to glean further information on the Owner. I cannot directly obtain the conclusion that the new LAK is  loaded on the same TPM as the IAK, but I am able to make an important conclusion regarding the sequence that the Owner executes. That is, the sequence \verb|s| is a supersequence of the correct steps of the Owner. A list is a supersequence of another list if all the elements of the second list occur, in order, in the first --- the elements need not occur consecutively. I define a cascading collection of recursive functions to determine whether a given sequence of commands is a supersequence of \verb|steps1to5_Owner|. 
Then using the initial assumptions regarding the Owner and the CA (i.e., all lines except the last in Figure \ref{fig:lak_goal}), I prove that the Owner's unknown sequence \verb|s| satisfies this function. This proof relies on the particular structure of the certification request as it is required by the CA's steps. In order to produce a message with this structure, specific commands must be executed in a specific order.
%The proof proceeds as follows. Use the match statement in \verb|steps_CA| to determine the structure of the certification request. Then use the sequence in \verb|steps_CA| to learn that some keys in the certification request are in fact the inverses of other keys. This resulting message's structure is such that it must have been constructed Then use this resulting message structure and the restrictions on the Owner's initial state pair to
%Learn about csr from steps_ca. To construct a final message which 


The overall proof hinges on this conclusion and proceeds fairly naturally going forward. Recall our musings in scenario 1 which reason that the LAK and IAK must be loaded on the same TPM if one certifies the LAK with the IAK. Therefore, our next step is to demonstrate that the Owner must have executed \verb|TPM2_Certify| on the public LAK and private IAK. 
Using the conclusion obtained above it is trivial to prove that this command is contained within the Owner's sequence \verb|s|.
I use the function \verb|command_in_sequence| to accurately describe this situation because all of the command inputs must match exactly. 
Then finally I prove one last set of intermediate lemmas that authoritatively state that the LAK and IAK must be loaded on the same TPM for one to execute any sequence that contains this command.
The composition of these small proofs leads to our end goal shown in Figure \ref{fig:lak_goal}. Due to the extra parameters and criteria of the \verb|steps_CA| function, the proof statement precisely states that the inverse of the key contained in the IAK certificate and the inverse of the key contained in the new certificate are loaded on the same TPM which is precisely the assertion described by Assurance B. 
Therefore, we have now confirmed that Assurance B is in fact guaranteed by the protocol when we assume only that the CA is trusted to execute its steps correctly. 


In conclusion, this procedure uses the previously-provisioned IAK to prove that the new LAK is loaded on the same TPM as itself and thus contained in the device identified by the IAK certificate. Therefore, when issuing the LAK certificate, the CA should use the same device identifying information as that from the IAK certificate's Subject field. To briefly summarize our results, Assurance A is guaranteed by the attribute check performed by the CA, and Assurance B is guaranteed by the IAK's \verb|Restricted| attribute and the signed \verb|TPM2B_Attest| structure.






\newpage
\section{OEM Creation of IAK Certificate based on EK Certificate}
The TCG's specification claims that the following procedure provides these assurances: (A) the IAK has good attributes, (B) the IAK is  loaded on the same TPM as the EK, and (C) the EK certificate is valid. These assurances correspond exactly with the Proof of Residency line A in Figure \ref{fig:cert_rel}. 
\begin{enumerate}[itemsep=0pt,parsep=0pt,partopsep=0pt]
  \setcounter{enumi}{-1}
  \item The OEM creates and loads the IAK
  \item The OEM builds the CSR containing:
  \begin{enumerate}[topsep=0pt, itemsep=0pt,parsep=0pt,partopsep=0pt]
    \item Device identity information including the device model and serial
    number
    \item The EK certificate
    \item The IAK public area
  \end{enumerate}
  \item The OEM takes a signature hash of the CSR
  \item The OEM signs the resulting hash digest with the IAK
  \item The OEM sends the CSR paired with the signed hash to the CA
  \item The CA verifies the received data by checking:
  \begin{enumerate}[topsep=0pt, itemsep=0pt,parsep=0pt,partopsep=0pt]
    \item The hash digest against the CSR
    \item The signature on the hash digest with the IAK public key
    \item The signature on the EK certificate with the public key of the TPM Manufacturer's CA
    \item The attributes of the IAK
  \end{enumerate}
  \item If all of the checks succeed, the CA issues a challenge blob to the OEM by:
  \begin{enumerate}[topsep=0pt, itemsep=0pt,parsep=0pt,partopsep=0pt]
    \item Calculating the cryptographic name of the IAK
    \item Generating a nonce
    \item Building the encrypted credential structure
  \end{enumerate}
  \item The OEM releases the secret nonce by decrypting the challenge blob and verifying the name of the IAK
  \item The CA checks the returned nonce against the one generated in Step 6b
  \item If the check succeeds, the CA issues the IAK certificate to the OEM
\end{enumerate}

This procedure is very similar to the one described in the previous section. In fact, nearly all steps of the LAK provisioning protocol---specifically all steps except for Steps 1 and 6c---are approximately included in this procedure. Therefore, this section need only expound on the details which differ from the previous verification process. 

Modeling this protocol begins by defining parameters for each the OEM and CA. The OEM has its IAK, EK, EK certificate, and device identifying information and the CA has its own key, the public key of the TPM Manufacturer's CA, and a secret nonce.
\begin{figure}[h]
\begin{lstlisting}[language=Coq]
(* OEM parameters *)
Parameter pubIAK : pubKey.
Parameter pubEK : pubKey.
Parameter certEK : signedCert.
Parameter devInfo : deviceInfoType.

(* CA parameters *)
Parameter pubCA : pubKey.
Parameter pubTM : pubKey.
Parameter nonce : randType.
\end{lstlisting}
\caption{Parameters of IAK Provisioning Procedure}
\end{figure}
The procedure may be regarded as the composition of four parts: the OEM's initial steps (i.e., Steps 0-4) followed by the CA's initial steps (i.e., Steps 5-6) followed the OEM's final step (i.e., Step 7) followed by the CA's final steps (i.e., Steps 8-9). The initial steps of the OEM and the OEM's CA are constructed similarly to the steps of the Owner and the Owner's CA respectively. In any case, the OEM's final step is naturally constructed as a simple function; first the OEM waits to receive a challenge blob from the CA, then the OEM executes Step 7 of the procedure. As a result of these definitions, the CA's final steps are implicit in the proof statements and do not require an explicit definition. 

Now proving Assurance A is trivial and proceeds identically to the corresponding proof in the previous section.
Proving Assurance C is in fact quite similar to this proof as well. The comand \verb|CheckCert (Cert k0 id0 k_ca') pubTM| in the CA's function corresponds with Step 5c of the provisioning procedure (see that the CA's function binds the EK to the variable \verb|k0| and the EK certificate to the message \verb|Cert k0 id0 k_ca'|). Then it is clear to see that successful execution of this command directly implies that the EK certificate is valid.

\vspace{2em}
\begin{figure}[h!]
\begin{lstlisting}[language=Coq]
Definition steps_CA (msg : message) (ident : identifier) (ek iak : pubKey) (cert : signedCert) : Prop :=
  match msg with
  | (pair (TCG_CSR_IDevID (Device_info id) (Cert k0 id0 k_ca') k) (signature m k')) =>
        ident = (Device_info id) /\ ek = k0 /\ iak = k /\ cert = (Cert k0 id0 k_ca') /\
        seq_execute   (iniTPM_CA, inferFrom msg ++ ini_CA)
                      (CheckHash 
                          msg 
                         (TCG_CSR_IDevID (Device_info id) (Cert k0 id0 k_ca') k) ;;
                       CheckSig 
                         (signature m k') 
                          k ;;
                       CheckCert 
                         (Cert k0 id0 k_ca') 
                          pubTM ;;
                       CheckAttributes 
                          k 
                          Restricting Signing NonDecrypting Fixing ;;
                       TPM2_Hash 
                         (publicKey k);;
                       TPM2_MakeCredential 
                         (hash (publicKey k))
                          nonce
                          k0 ;;
                       Done)
                      (hash (publicKey k) ::iniTPM_CA, 
                       encryptedCredential (hash (publicKey k)) nonce k0 :: hash (publicKey k) 
                       ::inferFrom msg ++ ini_CA)
  | _ => False
  end.


Definition step7_OEM (msg : message) : sequence :=
TPM2_ActivateCredential 
    msg 
    privEK 
    privIAK ;;
Done.
\end{lstlisting}
\caption{Model of IAK Provisioning Procedure}
\label{fig:iak_model}
\end{figure}


\vspace{3em}
Let us attempt formal verification of Assurance B first under the conditions of scenario 1: the OEM and the CA are both trusted to execute their steps correctly. I implement the same strategy as before, that is, I aim to show that the private IAK and private EK are contained in the OEM's minimal initial \verb|tpm_state|. I build a minimal initial state pair for the composition of \verb|steps1to4_OEM| and \verb|step7_OEM| using the following intuition: (i) the EK certificate and public IAK are known to the Owner because they are included in the CSR and (ii) the private IAK and private EK are loaded on the same TPM because the IAK is credentialed by the EK. The two required proofs to determine minimality proceed similarly to the corresponding proof in the previous section.

%This intuition is used to guide the proof of the lower bound property. While the sufficiency property uses the preconditions that the CA decides to issue the IAK certificate and that the EK has good attributes. The completion of these proofs confirm that Assurance B is in fact guaranteed by the protocol when we assume that both the OEM and the CA are trusted to execute their steps correctly.

Let us now attempt formal verification of this same goal under the conditions of scenario 2: only the CA is trusted to execute its steps correctly. 
I describe the OEM and its characteristics as a series of assumptions: the OEM executes some unknown sequence of commands \verb|s1|, this sequence produces some message \verb|msg| in the OEM's intermediate \verb|state|, the OEM's initial \verb|tpm_state| may only contain private keys, the OEM's initial \verb|state| may only contain public keys and certificates, the CA executes its prescribed sequence on the message \verb|msg| and sends the challenge blob \verb|encryptedCredential (hash (publicKey iak)) g ek| to the OEM, the OEM executes some unknown sequence of commands \verb|s2|, this sequence releases the secret nonce value \verb|g| into the OEM's final \verb|state|. The CA's final steps are implicit in the fact that the nonce in the challenge blob is the same as the nonce in the OEM's final \verb|state|. 
\begin{figure}[h]
\begin{lstlisting}[language=Coq]
Theorem iak_and_ek_in_TPM : forall s2 s1 iniTPM ini midTPM mid finTPM fin msg ident ek iak cert g,
  seq_execute   (iniTPM, ini) s1 (midTPM, mid) -> 
  In msg mid ->
  (forall m', needsGeneratedTPM m' -> ~ In m' iniTPM) ->
  (forall m', needsGenerated m' -> ~ In m' ini) ->
  steps_CA msg ident ek iak cert ->
  seq_execute   (midTPM, inferFrom (encryptedCredential (hash (publicKey iak)) g ek) ++ mid) 
                 s2 
                (finTPM, fin) ->
  In (randomNum g) fin ->
  In (privateKey (pubToPrivKey iak)) iniTPM /\ In (privateKey (pubToPrivKey ek)) iniTPM.
\end{lstlisting}
\caption{Final Verification Goal for IAK Provisioning Procedure}
\label{fig:iak_goal}
\end{figure}



In order to complete this verification, no further knowledge of the OEM's initial sequence \verb|s1| is needed.
On the other hand, the OEM's final sequence \verb|s2| must be analyzed.
I prove that \verb|s2| is a supersequence of the correct final steps of the OEM (i.e., \verb|step7_OEM|). This proof relies on two important, related properties of sequential execution. The first being that one cannot produce a random number in its final \verb|state| without an encrypted credential containing that number in its initial \verb|state|. The next property states approximately the converse of the first, that is, one cannot produce an encrypted credential containing a particular random number in its final \verb|state| without that number or an encrypted credential containing that number in its initial \verb|state|.
Our next step is simply to demonstrate that the OEM must in fact have executed \verb|TPM2_ActivateCredential| on the challenge blob using the private EK and private IAK. This result is a direct consequence of the supersequence conclusion. Then we need only verify that execution of a sequence containing this command implies that the EK and IAK are loaded on the same TPM. And the composition of these small proofs leads to our end goal shown in Figure 
\ref{fig:iak_goal}.
Due to the extra parameters and criteria of the
\verb|steps_CA| function, the proof statement precisely states that the the inverse of the key contained in
the EK certificate and the inverse of the key contained in the new certificate are loaded on the
same TPM which is precisely the assertion described by Assurance B.

In conclusion, this procedure uses the previously-certified EK to prove that the new IAK is
loaded on the same TPM as itself. When issuing the IAK certificate, the CA uses the device identifying information from the recieved CSR. 
%should use the device identify information from the CSR in the Subject field of the IAK certificate.
To briefly summarize, Assurance A
is guaranteed by the attribute check performed by the CA, Assurance B is guaranteed by the EK's
\verb|Restricted| attribute and the secret contained in the challenge blob, and Assurance C is guaranteed by the signature check on the EK certificate performed by the CA.

%Although, the IAK certificate identifies a device and is in fact intended to provide definitive evidence that a key belongs to a specific device, the IAK provisioning procedure provides no assurances on device-residency. This recommended procedure does not even include any steps where the CA attempts any form of device identity verification.  Instead the CA fully trusts the OEM to have provided the correct device identifying information. Even the attestation variation of this procedure does not attempt this verification and instead only checks that the device is running appropriate and safe firmware. A fix to this would be for the PCR data to measure the device's serial and model numbers so that each device has a unique hash value and hence the identity can be uniquely determined and proven that the TPM is on that exact device. 





This analysis has inadvertently revealed a potential shortcoming of the IAK provisioning procedure: it provides no assurance regarding device-residency.
The IAK certificate is intended to provide definitive evidence that a key belongs to a specific device, yet the procedure makes no guarantees that the IAK is actually on the device represented by that identity.
Proving an IAK belongs to a specific device requires first binding the IAK to a specific TPM using the EK and then binding the TPM to a specific device \cite{DevIDSpec-TCG}. 
But the CA makes no attempt at performing the second part of this process. Instead the CA fully trusts the OEM to have provided the correct device identifying information. 
Even the attestation variation of the IAK provisioning procedure does not attempt this verification. This variation uses a quote over selected PCRs to verify only that the device is running the appropriate, trusted firmware. All devices with the same model number have an identical golden hash value which represents its expected PCR digest result.
Although this check guarantees that the TPM containing the IAK is on a specific \textit{type} of device (i.e., a device with that model number), it does not guarantee that the TPM is on a specific device. 

I believe that a small change to the attestation variation of the IAK provisioning procedure can overcome this shortcoming. The TPM should take a measurement which contains the device's serial number. In this way. each individual device has a unique golden hash value, and thus its identity can be uniquely determined.
Due to the measurement integrity guaranteed by the TPM's PCR mechanism, this modification provides definitive evidence that a key belongs to a specific 
device, the exact result which an IAK certificate is intented to have.
%the IAK certificate is intended to have this exact result. 
%this result is exactly what the IAK certificate is intended to provide.
The major downfall of this solution is that it requires a very large database of these golden hash values. Therefore, it may be unrealistic and even unnecessary to implement this change in circumstances where a precise device identity is not required.
% stronger device binding is not required. 
But this solution may still be useful in high security systems where it is imperative to know the true identity of a device.



